\documentclass{article}

\usepackage{polyglossia}
\setmainlanguage{italian}
\setotherlanguage{japanese}
\newfontfamily\japanesefont{BIZ UDGothic}

\usepackage[margin=1in]{geometry}

\usepackage{multirow,longtable}
\renewcommand{\arraystretch}{1.5}

\title{Grammatica}
\author{Mattia Martelli}
\date{}

\usepackage{hyperref}
\hypersetup{
    unicode=true,
    bookmarksnumbered=true,
    bookmarksopen=false,
    hidelinks,
    pdfauthor={Mattia Martelli},
    pdftitle={Grammatica}
}

\let\ita\textitalian
\let\jap\textjapanese
\newcommand{\itabf}[1]{\ita{\textbf{#1}}}
\newenvironment{tabVoc}{\begin{center}\begin{japanese}\begin{longtable}{|c|c|c|}}{\end{longtable}\end{japanese}\end{center}}
\newenvironment{tabAdj}{\begin{center}\begin{japanese}\begin{longtable}{|c|c|c|c|}}{\end{longtable}\end{japanese}\end{center}}

\begin{document}

    \maketitle

    \tableofcontents
    
    \section{Stato dell'essere}

        \subsection{Positivo}

            \begin{tabVoc}
                \hline
                \multicolumn{3}{|c|}{\itabf{Vocaboli}}\\
                \hline
                \itabf{Kanji} & \itabf{Pronuncia} & \itabf{Significato}\\
                \hline\hline
                人 & ひと & \ita{Persona}\\
                \hline
                学生 & がく・せい & \ita{Studente}\\
                \hline
                元気 & げん・き & \ita{In salute, vivace}\\
                \hline
            \end{tabVoc}

            Per esprimere l'esistenza di qualcosa si attacca in fondo al nome oppure al na-aggettivo il kana \jap{だ}. Questo può però
            essere implicitato, soprattutto in un contesto informale.

            Esempi:
            \begin{enumerate}
                \item \jap{人だ}: è persona;
                \item \jap{学生だ}: è studente;
                \item \jap{元気だ}: è in salute;
                    \begin{enumerate}
                        \item \jap{元気?}: Stai bene?
                        \item \jap{元気}: Sto bene.
                    \end{enumerate}
            \end{enumerate}

        \subsection{Negativo}

            \begin{tabVoc}
                \hline
                \multicolumn{3}{|c|}{\itabf{Vocaboli}}\\
                \hline
                \itabf{Kanji} & \itabf{Pronuncia} & \itabf{Significato}\\
                \hline\hline
                学生 & がく・せい & \ita{Studente}\\
                \hline
                友達 & とも・だち & \ita{Amico}\\
                \hline
                元気 & げん・き & \ita{In salute, vivace}\\
                \hline
            \end{tabVoc}

            Per esprimere la non esistenza di qualcosa si attacca in fondo al nome oppure al na-aggettivo \jap{じゃない}.

            Esempi:
            \begin{enumerate}
                \item \jap{学生じゃない}: non è studente;
                \item \jap{友達じゃない}: non è amico;
                \item \jap{元気じゃない}: non è in salute.
            \end{enumerate}

        \subsection{Passato}

            \begin{tabVoc}
                \hline
                \multicolumn{3}{|c|}{\itabf{Vocaboli}}\\
                \hline
                \itabf{Kanji} & \itabf{Pronuncia} & \itabf{Significato}\\
                \hline\hline
                学生 & がく・せい & \ita{Studente}\\
                \hline
                友達 & とも・だち & \ita{Amico}\\
                \hline
                元気 & げん・き & \ita{In salute, vivace}\\
                \hline
            \end{tabVoc}

            Per esprimere la passata esistenza di qualcosa si attacca in fondo al nome oppure al na-aggettivo \jap{だつた}. Per il passato
            negativo, si sostituisce il kana \jap{い} da \jap{じゃない} con \jap{かつた}.

            Esempi:
            \begin{enumerate}
                \item \jap{学生だつた}: era studente;
                \item \jap{友達じゃなかつた}: non era amico;
                \item \jap{元気じゃなかつた}: non era in salute.
            \end{enumerate}

    \section{Particelle}

        \subsection{La particella di argomento \jap{は}}

            \begin{tabVoc}
                \hline
                \multicolumn{3}{|c|}{\itabf{Vocaboli}}\\
                \hline
                \itabf{Kanji} & \itabf{Pronuncia} & \itabf{Significato}\\
                \hline\hline
                学生 & がく・せい & \ita{Studente}\\
                \hline
                / & うん & \ita{Sì (informale)}\\
                \hline
                / & ううん & \ita{No (informale)}\\
                \hline
                明日 & あした & \ita{Domani}\\
                \hline
                今日 & きょう & \ita{Oggi}\\
                \hline
                試験 & しけん & \ita{Esame}\\
                \hline
            \end{tabVoc}

            La particella \jap{は}, pronuciata \jap{わ}, si usa per definire l'argomento della frase.

            \subsubsection*{Primo esempio}

                \begin{itemize}
                    \item \jap{ボブ:アリスは学生?} (Bob: È Alice (tu) una studentessa?)
                    \item \jap{アリス:うん、学生。} (Alice: Sì, (lo) sono.)
                \end{itemize}

                Bob indica che la domanda riguarda Alice, che a questo punto non deve ripetere l'argomento per rispondere.

            \subsubsection*{Secondo esempio}

                \begin{itemize}
                    \item \jap{ボブ:ジョンは明日?} (Bob: È John domani?)
                    \item \jap{アリス:ううん、明日じゃない。} (Alice: No, non domani.)
                \end{itemize}

                La conversazione è corretta, per quanto insensata senza contesto, il quale è stato implicitato.

            \subsubsection*{Terzo esempio}

                \begin{itemize}
                    \item \jap{アリス:今日は試験だ。} (Alice: L'esame è oggi.)
                    \item \jap{ボブ:ジョンは?} (Bob: Riguardo John?)
                    \item \jap{アリス:ジョンは明日} (Alice: John è domani.)
                \end{itemize}

                La particella è molto generica, in quanto può riferirsi ad argomenti anche di altre frasi. Infatti, l'ultima frase,
                anche se riferita all'esame di John, non presenta nemmeno la parola ``esame''.

        \subsection{La particella di argomento inclusiva \jap{も}}

            \begin{tabVoc}
                \hline
                \multicolumn{3}{|c|}{\itabf{Vocaboli}}\\
                \hline
                \itabf{Kanji} & \itabf{Pronuncia} & \itabf{Significato}\\
                \hline\hline
                学生 & がく・せい & \ita{Studente}\\
                \hline
                / & うん & \ita{Sì (informale)}\\
                \hline
                / & ううん & \ita{No (informale)}\\
                \hline
                / & でも & \ita{Ma}\\
                \hline
            \end{tabVoc}

            La particella \jap{も} è molto simile alla precedente \jap{は} e si usa per introdurre un ulteriore argomento nella
            conversazione.

            \subsubsection*{Primo esempio}

                \begin{itemize}
                    \item \jap{ボブ:アリスは学生?} (Bob: È Alice (tu) una studentessa?)
                    \item \jap{アリス:うん、トムも学生。} (Alice: Sì, ed anche Tom è uno studente.)
                \end{itemize}

                La particella deve essere usata coerentemente con la risposta: avrebbe poco senso dire ``sì, ed anche Tom non lo è''.

            \subsubsection*{Secondo esempio}

                \begin{itemize}
                    \item \jap{ボブ:アリスは学生?} (Bob: È Alice (tu) una studentessa?)
                    \item \jap{アリス:うん、でもトムは学生じゃない。} (Alice: Sì, ma Tom non è uno studente.)
                \end{itemize}

                In questo caso si usa \jap{は} in quanto la risposta muta argomento.

            \subsubsection*{Terzo esempio}

                \begin{itemize}
                    \item \jap{ボブ:アリスは学生?} (Bob: È Alice (tu) una studentessa?)
                    \item \jap{アリス:ううん、トムも学生じゃない。} (Alice: No, e neanche Tom è uno studente.)
                \end{itemize}

                Essendo la risposta coerente, si usa \jap{も}.

        \subsection{La particella identificativa \jap{が}}

            \begin{tabVoc}
                \hline
                \multicolumn{3}{|c|}{\itabf{Vocaboli}}\\
                \hline
                \itabf{Kanji} & \itabf{Pronuncia} & \itabf{Significato}\\
                \hline\hline
                学生 & がく・せい & \ita{Studente}\\
                \hline
                誰 & だれ & \ita{Chi}\\
                \hline
                私 & わたし & \ita{Me, me stesso, io}\\
                \hline
            \end{tabVoc}

            Questa particella viene usata quando non si sa l'argomento e lo si vuole identificare. Viene anche detta la particella
            soggetto. Viene inoltre utilizzata per identificare una propiretà specifica, mentre \jap{は} viene utilizzata per introdurre nuovi
            argomenti.

            \subsubsection*{Primo esempio}

                \begin{itemize}
                    \item \jap{ボブ:誰が学生?} (Bob: Chi è colui che è studente?)
                    \item \jap{アリス:ジョンが学生。} (Alice: John è lo studente.)
                \end{itemize}

                Se fosse stata usata la particella di argomento nella risposta il senso sarebbe stato che John è uno studente, ma non
                necessariamente quello richiesto.

            \subsubsection*{Secondo esempio}

                \begin{enumerate}
                    \item \jap{誰が学生?} (Chi è colui che è studente?)
                    \item \jap{学生は誰?} ((Lo) studente è chi?)
                \end{enumerate}

                La prima frase cerca di identificare uno studente preciso, mentre la seconda parla dello studente. Non avrebbe senso
                sostituire \jap{が} con \jap{は} nella prima frase, in quanto questa diventerebbe ``è chi lo studente?''.

            \subsubsection*{Terzo esempio}

                \begin{enumerate}
                    \item \jap{私は学生}: io (sono) studente;
                    \item \jap{私が学生}: io (sono) studente.
                \end{enumerate}

                Entrambe possono essere tradotte nello stesso modo, ma il significato è differente: la prima può essere vista come
                ``parlando di me, sono uno studente'', mentre la seconda può essere vista come ``io sono colui che è uno studente''.
        
    \section{Aggettivi}

        \subsection{Na-aggettivi}

            \begin{tabAdj}
                \hline
                \multicolumn{4}{|c|}{\itabf{Vocaboli}}\\
                \hline
                \itabf{Kanji} & \itabf{Pronuncia} & \itabf{Tipo} & \itabf{Significato}\\
                \hline\hline
                静か & しず・か & \ita{Na-aggettivo} & \ita{Tranquillo}\\
                \hline
                人 & ひと & \ita{Nome} & \ita{Persona}\\
                \hline
                / & きれい & \ita{Na-aggettivo} & \ita{Carino, pulito}\\
                \hline
                友達 & とも・だち & \ita{Nome} & \ita{Amico}\\
                \hline
                新設 & しん・せつ & \ita{Na-aggettivo} & \ita{Gentile}\\
                \hline
                魚 & さかな & \ita{Nome} & \ita{Pesce}\\
                \hline
                好き & す・き & \ita{Na-aggettivo} & \ita{Piacevole, desiderabile}\\
                \hline
                肉 & にく & \ita{Nome} & \ita{Carne}\\
                \hline
                野菜 & や・さい & \ita{Nome} & \ita{Verdura}\\
                \hline
            \end{tabAdj}

            I na-aggettivi seguono le medesime regole di coniugazione dei nomi. La differenza sta nel fatto che i na-aggettivi possono
            modificare un nome che li segue intramezzando i due con \jap{な}.

            \subsubsection*{Primo esempio}

                \begin{enumerate}
                    \item \jap{静かな人}: persona tranquilla;
                    \item \jap{きれいな人}: persona carina.
                \end{enumerate}

            \subsubsection*{Secondo esempio}

                \begin{enumerate}
                    \item \jap{友達は新設}: amico è gentile;
                    \item \jap{友達は新設な人だ}: amico è persona gentile.
                \end{enumerate}

            \subsubsection*{Terzo esempio}

                \begin{enumerate}
                    \item \jap{ボブは魚が好きだ}: a Bob piace il pesce;
                    \item \jap{ボブは魚が好きじゃない}: a Bob non piace il pesce;
                    \item \jap{ボブは魚が好きだつた}: a Bob piaceva il pesce;
                    \item \jap{ボブは魚が好きじゃなかつた}: a Bob non piaceva il pesce.
                \end{enumerate}

                La frase riguarda Bob, come evidenzia la particella \jap{は}, e ``pesce'' specifica ciò che piace a Bob, tramite
                la particella \jap{が}.

            \subsubsection*{Quarto esempio}

                \begin{enumerate}
                    \item \jap{魚が好きな人}: persona a cui piace il pesce;
                    \item \jap{魚が好きじゃない人}: persona a cui non piace il pesce;
                    \item \jap{魚がすきだつた人}: persona a cui piaceva il pesce;
                    \item \jap{魚が好きじゃなかつた人}: persona a cui non piaceva il pesce.
                \end{enumerate}

                Si usa \jap{な} solo per il presente positivo, mentre si cognuga semplicemente in tutti gli altri casi.

            \subsubsection*{Quinto esempio}

                \begin{enumerate}
                    \item \jap{魚が好きじゃない人は、肉が好きだ}: alle persone a cui non piace il pesce, piace la carne;
                    \item \jap{魚が好きな人は、野菜も好きだ}: alle persone a cui piace il pesce piacciono anche le verdure.
                \end{enumerate}

        \subsection{Gli i-aggettivi}

            \begin{tabAdj}
                \hline
                \multicolumn{4}{|c|}{\itabf{Vocaboli}}\\
                \hline
                \itabf{Kanji} & \itabf{Pronuncia} & \itabf{Tipo} & \itabf{Significato}\\
                \hline\hline
                嫌い & きら・い & \ita{Na-aggettivo} & \ita{Di cattivo gusto, odioso}\\
                \hline
                食べ物 & た・べ・もの & \ita{Nome} & \ita{Cibo}\\
                \hline
                / & おいしい & \ita{I-aggettivo} & \ita{Gustoso}\\
                \hline
                高い & たか・い & \ita{I-aggettivo} & \ita{Alto, costoso}\\
                \hline
                / & ビル & \ita{Nome} & \ita{Edificio}\\
                \hline
                値段 & ね・だん & \ita{Nome} & \ita{Prezzo}\\
                \hline
                / & レストラン & \ita{Nome} & \ita{Ristorante}\\
                \hline
                / & あまり/あんまり & \ita{Nome} & \ita{Non molto (usato col negativo)}\\
                \hline
                好き & す・き & \ita{Na-aggettivo} & \ita{Piacevole, desiderabile}\\
                \hline
                / & いい & \ita{I-aggettivo} & \ita{Buono}\\
                \hline
            \end{tabAdj}

            Gli i-aggettivi sono così chiamati poiché terminano con il kana \jap{い}. Anche alcuni na-aggetivi terminano con lo
            stesso suono, ma è incluso nella scrittura kanji, eccezzion fatta per alcuni rari casi, rendendo relativamente
            semplice la distinzione.

            A differenza dei na-aggettivi, non bisogna mettere il kana \jap{な} per modificare un nome. Inoltre, non bisogna
            attaccare la particella \jap{だ} agli i-aggettivi.
            
            Per il negativo, sostituire il kana \jap{い} finale con \jap{くない}. Per il passato, sostituire il kana con
            \jap{かつた}. Per il passato negativo, prima fare il negativo e poi fare il passato.

            \subsubsection*{Primo esempio}

                \begin{enumerate}
                    \item \jap{嫌いな食べ物}: cibo sgradevole;
                    \item \jap{おいしい食べ物}: cibo gustoso.
                \end{enumerate}

            \subsubsection*{Secondo esempio}

                \begin{enumerate}
                    \item \jap{高いビル}: edificio alto;
                    \item \jap{高くないビル}: edificio non alto;
                    \item \jap{高かつたビル}: edificio che era alto;
                    \item \jap{高くなかつたビル}: edificio che non era alto.
                \end{enumerate}

            \subsubsection*{Terzo esempio}

                \begin{enumerate}
                    \item \jap{値段が高いレストランはあまり好きじゃない}: non piacciono molto i ristoranti costosi.
                \end{enumerate}

                In questo esempio la clausola descrittiva \jap{値段が高い} modifica direttamente \jap{レストラン}.

\end{document}