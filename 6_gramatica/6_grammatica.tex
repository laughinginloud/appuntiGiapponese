\documentclass{article}

\usepackage{polyglossia}
\setmainlanguage{italian}
\setotherlanguage{japanese}
\newfontfamily\japanesefont{BIZ UDGothic}

\usepackage[margin=1in]{geometry}

\usepackage{multirow,longtable}
\renewcommand{\arraystretch}{1.5}

\title{Grammatica}
\author{Mattia Martelli}
\date{}

\usepackage{hyperref}
\hypersetup{
    unicode=true,
    bookmarksnumbered=true,
    bookmarksopen=false,
    hidelinks,
    pdfauthor={Mattia Martelli},
    pdftitle={Grammatica}
}

\let\ita\textitalian
\let\jap\textjapanese
\newcommand{\itabf}[1]{\ita{\textbf{#1}}}
\newenvironment{tabVoc}{\begin{center}\begin{japanese}\begin{longtable}{|c|c|c|}}{\end{longtable}\end{japanese}\end{center}}

\begin{document}

    \maketitle

    \tableofcontents
    
    \section{Stato dell'essere}

        \subsection{Positivo}

            \begin{tabVoc}
                \hline
                \multicolumn{3}{|c|}{\itabf{Vocaboli}}\\
                \hline
                \itabf{Kanji} & \itabf{Pronuncia} & \itabf{Significato}\\
                \hline\hline
                人 & ひと & \ita{Persona}\\
                \hline
                学生 & がく・せい & \ita{Studente}\\
                \hline
                元気 & げん・き & \ita{In salute, vivace}\\
                \hline
            \end{tabVoc}

            Per esprimere l'esistenza di qualcosa si attacca in fondo al nome oppure al na-aggettivo il kana \jap{だ}. Questo può però
            essere implicitato, soprattutto in un contesto informale.

            Esempi:
            \begin{enumerate}
                \item \jap{人だ}: è persona;
                \item \jap{学生だ}: è studente;
                \item \jap{元気だ}: è in salute;
                    \begin{enumerate}
                        \item \jap{元気?}: Stai bene?
                        \item \jap{元気}: Sto bene.
                    \end{enumerate}
            \end{enumerate}

        \subsection{Negativo}

            \begin{tabVoc}
                \hline
                \multicolumn{3}{|c|}{\itabf{Vocaboli}}\\
                \hline
                \itabf{Kanji} & \itabf{Pronuncia} & \itabf{Significato}\\
                \hline\hline
                学生 & がく・せい & \ita{Studente}\\
                \hline
                友達 & とも・だち & \ita{Amico}\\
                \hline
                元気 & げん・き & \ita{In salute, vivace}\\
                \hline
            \end{tabVoc}

            Per esprimere la non esistenza di qualcosa si attacca in fondo al nome oppure al na-aggettivo \jap{じゃない}.

            Esempi:
            \begin{enumerate}
                \item \jap{学生じゃない}: non è studente;
                \item \jap{友達じゃない}: non è amico;
                \item \jap{元気じゃない}: non è in salute.
            \end{enumerate}

        \subsection{Passato}

            \begin{tabVoc}
                \hline
                \multicolumn{3}{|c|}{\itabf{Vocaboli}}\\
                \hline
                \itabf{Kanji} & \itabf{Pronuncia} & \itabf{Significato}\\
                \hline\hline
                学生 & がく・せい & \ita{Studente}\\
                \hline
                友達 & とも・だち & \ita{Amico}\\
                \hline
                元気 & げん・き & \ita{In salute, vivace}\\
                \hline
            \end{tabVoc}

            Per esprimere la passata esistenza di qualcosa si attacca in fondo al nome oppure al na-aggettivo \jap{だつた}. Per il passato
            negativo, si sostituisce il kana \jap{い} da \jap{じゃない} con \jap{かつた}.

            Esempi:
            \begin{enumerate}
                \item \jap{学生だつた}: era studente;
                \item \jap{友達じゃなかつた}: non era amico;
                \item \jap{元気じゃなかつた}: non era in salute.
            \end{enumerate}

    \section{Particelle}

        \subsection{La particella di argomento \jap{は}}

            \begin{tabVoc}
                \hline
                \multicolumn{3}{|c|}{\itabf{Vocaboli}}\\
                \hline
                \itabf{Kanji} & \itabf{Pronuncia} & \itabf{Significato}\\
                \hline\hline
                学生 & がく・せい & \ita{Studente}\\
                \hline
                / & うん & \ita{Sì (informale)}\\
                \hline
                / & ううん & \ita{No (informale)}\\
                \hline
                明日 & あした & \ita{Domani}\\
                \hline
                今日 & きょう & \ita{Oggi}\\
                \hline
                試験 & しけん & \ita{Esame}\\
                \hline
            \end{tabVoc}

            La particella \jap{は}, pronuciata \jap{わ}, si usa per definire l'argomento della frase.

            \subsubsection*{Primo esempio}

                \begin{itemize}
                    \item \jap{ボブ:アリスは学生?} (Bob: È Alice (tu) una studentessa?)
                    \item \jap{アリス:うん、学生。} (Alice: Sì, (lo) sono.)
                \end{itemize}

                Bob indica che la domanda riguarda Alice, che a questo punto non deve ripetere l'argomento per rispondere.

            \subsubsection*{Secondo esempio}

                \begin{itemize}
                    \item \jap{ボブ:ジョンは明日?} (Bob: È John domani?)
                    \item \jap{アリス:ううん、明日じゃない。} (Alice: No, non domani.)
                \end{itemize}

                La conversazione è corretta, per quanto insensata senza contesto, il quale è stato implicitato.

            \subsubsection*{Terzo esempio}

                \begin{itemize}
                    \item \jap{アリス:今日は試験だ。} (Alice: L'esame è oggi.)
                    \item \jap{ボブ:ジョンは?} (Bob: Riguardo John?)
                    \item \jap{アリス:ジョンは明日} (Alice: John è domani.)
                \end{itemize}

                La particella è molto generica, in quanto può riferirsi ad argomenti anche di altre frasi. Infatti, l'ultima frase,
                anche se riferita all'esame di John, non presenta nemmeno la parola ``esame''.

        \subsection{La particella di argomento inclusiva \jap{も}}

            \begin{tabVoc}
                \hline
                \multicolumn{3}{|c|}{\itabf{Vocaboli}}\\
                \hline
                \itabf{Kanji} & \itabf{Pronuncia} & \itabf{Significato}\\
                \hline\hline
                学生 & がく・せい & \ita{Studente}\\
                \hline
                / & うん & \ita{Sì (informale)}\\
                \hline
                / & ううん & \ita{No (informale)}\\
                \hline
                / & でも & \ita{Ma}\\
                \hline
            \end{tabVoc}

            La particella \jap{も} è molto simile alla precedente \jap{は} e si usa per introdurre un ulteriore argomento nella
            conversazione.

            \subsubsection*{Primo esempio}

                \begin{itemize}
                    \item \jap{ボブ:アリスは学生?} (Bob: È Alice (tu) una studentessa?)
                    \item \jap{アリス:うん、トムも学生。} (Alice: Sì, ed anche Tom è uno studente.)
                \end{itemize}

                La particella deve essere usata coerentemente con la risposta: avrebbe poco senso dire ``sì, ed anche Tom non lo è''.

            \subsubsection*{Secondo esempio}

                \begin{itemize}
                    \item \jap{ボブ:アリスは学生?} (Bob: È Alice (tu) una studentessa?)
                    \item \jap{アリス:うん、でもトムは学生じゃない。} (Alice: Sì, ma Tom non è uno studente.)
                \end{itemize}

                In questo caso si usa \jap{は} in quanto la risposta muta argomento.

            \subsubsection*{Terzo esempio}

                \begin{itemize}
                    \item \jap{ボブ:アリスは学生?} (Bob: È Alice (tu) una studentessa?)
                    \item \jap{アリス:ううん、トムも学生じゃない。} (Alice: No, e neanche Tom è uno studente.)
                \end{itemize}

                Essendo la risposta coerente, si usa \jap{も}.
        
\end{document}