\documentclass{article}

\usepackage{polyglossia}
\setmainlanguage{italian}
\setotherlanguage{japanese}
\newfontfamily\japanesefont{BIZ UDGothic}

\usepackage[margin=1in]{geometry}

\usepackage{multirow, longtable}
\renewcommand{\arraystretch}{1.5}

\title{Vocaboli}
\author{Mattia Martelli}
\date{}

\usepackage{hyperref}
\hypersetup{
    unicode=true,
    bookmarksnumbered=true,
    bookmarksopen=false,
    hidelinks,
    pdfauthor={Mattia Martelli},
    pdftitle={Kanji}
}

\let\ita\textitalian
\newcommand{\itabf}[1]{\ita{\textbf{#1}}}

\begin{document}
    
    \maketitle

    %\tableofcontents

    \begin{center}
        \begin{japanese}
            \begin{longtable}{|c|c|c|}
                \hline
                \itabf{Vocabolo} & \itabf{Pronuncia} & \itabf{Significato}\\
                \hline\hline
                アメリカ人 & アメリカ・じん & \ita{Americano (persona)}\\
                \hline
                フランス人 & フランス・じん & \ita{Francese (persona)}\\
                \hline
                日本 & に・ほん & \ita{Giappone}\\
                \hline
                本 & ほん & \ita{Libro}\\
                \hline
                高い & たか・い & \ita{Alto, costoso}\\
                \hline
                学校 & がっ・こう & \ita{Scuola}\\
                \hline
                高校 & こう・こう & \ita{Scuola superiore}\\
                \hline
                小さい & ちい・さい & \ita{Piccolo}\\
                \hline
                大きい & おお・きい & \ita{Grande}\\
                \hline
                小学校 & しょう・がっ・こう & \ita{Scuola elementare}\\
                \hline
                中学校 & ちゅう・がっ・こう & \ita{Scuola media}\\
                \hline
                大学 & だい・がく & \ita{College, università}\\
                \hline
                小学生 & しょう・がく・せい & \ita{Studente di scuola elementare}\\
                \hline
                中学生 & ちゅう・がく・せい & \ita{Studente di scuola media}\\
                \hline
                大学生 & だい・がく・せい & \ita{Studente universitario}\\
                \hline
                国 & くに & \ita{Paese}\\
                \hline
                中国 & ちゅう・ごく & \ita{Cina}\\
                \hline
                中国人 & ちゅう・ごく・じん & \ita{Cinese (persona)}\\
                \hline
                日本語 & に・ほん・ご & \ita{Lingua giapponese}\\
                \hline
                中国語 & ちゅう・ごく・ご & \ita{Lingua cinese}\\
                \hline
                英語 & えい・ご & \ita{Lingua inglese}\\
                \hline
                フランス語 & フランス・ご & \ita{Lingua francese}\\
                \hline
                スペイン語 & スペイン・ご & \ita{Lingua spagnola}\\
                \hline
            \end{longtable}
        \end{japanese}
    \end{center}

\end{document}