\documentclass{article}

\usepackage{polyglossia}
\setmainlanguage{italian}
\setotherlanguage{japanese}
\newfontfamily\japanesefont{BIZ UDGothic}

\usepackage[landscape, margin=1in]{geometry}

\usepackage{multirow}
\renewcommand{\arraystretch}{1.5}

\title{Hiragana e katakana}
\author{Mattia Martelli}
\date{}

\usepackage{hyperref}
\hypersetup{
    unicode=true,
    bookmarksnumbered=true,
    bookmarksopen=false,
    hidelinks,
    pdfauthor={Mattia Martelli},
    pdftitle={Hiragana e katakana}
}

\begin{document}

    \maketitle

    \tableofcontents

    \section{Hiragana e katakana}

        \begin{center}
            \begin{japanese}
                \begin{tabular}{|c|c|c|c|c|c|c|c|c|c|c|}
                    \hline
                    \multirow{5}{*}{ん ン n}
                        & わ ワ wa & ら ラ ra & や ヤ ya & ま マ ma & は ハ ha & な ナ na & た タ ta & さ サ sa & か カ ka & あ ア a\\
                        \cline{2-11}
                        & ゐ ヰ wi & り リ ri & & み 三 mi & ひ ヒ hi & に 二 ni & ち チ chi & し シ shi & き キ ki & い イ i\\
                        \cline{2-11}
                        & & る ル ru & ゆ ユ yu & む ム mu & ふ フ fu & ぬ ヌ nu & つ ツ tsu & す ス su & く ク ku & う ウ u\\
                        \cline{2-11}
                        & ゑ ヱ we & れ レ re & & め メ me & へ ヘ he & ね ネ ne & て テ te & せ セ se & け ケ ke & え エ e\\
                        \cline{2-11}
                        & を ヲ o & ろ ロ ro & よ ヨ yo & も モ mo & ほ ホ ho & の ノ no & と ト to & そ ソ so & こ コ ko & お オ o\\
                    \hline
                \end{tabular}
            \end{japanese}
        \end{center}

        \textbf{Nota}: i kana relativi ai suoni \textjapanese{wi} e \textjapanese{we} sono oramai caduti in disuso nell'uso comune.
    
    \section{Dakuten e handakuten}

        \begin{center}
            \begin{japanese}
                \begin{tabular}{|c|c|c|c|c|}
                    \hline
                    ぱ パ pa & ば バ ba & だ ダ da & ざ ザ za & が ガ ga\\
                    \hline
                    ぴ ピ pi & ば バ bi & ぢ ヂ ji & じ ジ ji & ぎ ギ gi\\
                    \hline
                    ぷ プ pu & ぶ ブ bu & づ ヅ zu & ず ズ zu & ぐ グ gu\\
                    \hline
                    ぺ ペ pe & べ べ be & で デ de & ぜ ゼ ze & げ ゲ ge\\
                    \hline
                    ぽ ポ po & ぼ ボ bo & ど ド do & ぞ ゾ zo & ご ゴ go\\
                    \hline
                \end{tabular}
            \end{japanese}
        \end{center}

        \textbf{Nota}: \textjapanese{ぢ} e \textjapanese{づ} possono essere scritti in rōmaji anche come \textjapanese{dji} e \textjapanese{dzu}, rispettivamente.

    \section{Y\={o}on}

        \subsection{Hiragana e katakana}

            \begin{center}
                \begin{japanese}
                    \begin{tabular}{|c|c|c|c|c|c|c|c|}
                        \hline
                        りゃ リャ rya & みゃ ミャ mya & ひゃ ヒャ hya & にゃ ニャ nya & ちゃ チャ cha & しゃ シャ sha & きゃ キャ kya\\
                        \hline
                        りゅ リュ ryu & みゅ ミュ myu & ひゅ ヒュ hyu & にゅ ニュ nyu & ちゅ チュ chu & しゅ シュ shu & きゅ キュ kyu\\
                        \hline
                        りょ リョ ryo & みょ ミョ myo & ひょ ヒョ hyo & にょ ニョ nyo & ちょ チョ cho & しょ ショ sho & きょ キョ kyo\\
                        \hline
                    \end{tabular}
                \end{japanese}
            \end{center}

        \subsection{Dakuten e handakuten}

            \begin{center}
                \begin{japanese}
                    \begin{tabular}{|c|c|c|c|}
                        \hline
                        ぴゃ ピャ pya & びゃ ビャ bya & じゃ ジャ jya & ぎゃ ギャ gya\\
                        \hline
                        ぴゅ ピュ pyu & びゅ ビュ byu & じゅ ジュ jyu & ぎゅ ギュ gyu\\
                        \hline
                        ぴょ ピョ pyo & びょ ビョ byo & じょ ジョ jyo & ぎょ ギョ gyo\\
                        \hline
                    \end{tabular}
                \end{japanese}
            \end{center}

            \textbf{Nota}: \textjapanese{じゃ}, \textjapanese{じゅ} e \textjapanese{じょ} possono essere scritti in rōmaji
            anche come \textjapanese{ja}, \textjapanese{ju} e \textjapanese{jo}, rispettivamente.

    \section{Sokuon}

        Per raddoppiare una consonante si usa una versione rimpicciolita di \textjapanese{tsu} davanti alla consonante.
        Ad esempio, \textjapanese{ひと}, che si pronuncia \textjapanese{hito}, e \textjapanese{ひっと}, che si pronuncia \textjapanese{hitto}.

    \newpage

    \section{Ch\={o}on}

        Per allungare un suono vocalico in hiragana si aggiunge un kana dopo la vocale, secondo la seguente tabella.

        \begin{center}
            \begin{japanese}
                \begin{tabular}{|c|c|}
                    \hline
                    \textitalian{Suono vocalico} & \textitalian{Kana}\\
                    \hline\hline
                    /a/ & あ\\
                    \hline
                    /i/, /e/ & い\\
                    \hline
                    /u/, /o/ & う\\
                    \hline
                \end{tabular}
            \end{japanese}
        \end{center}

        Ad esempio, \textjapanese{せんせい} si pronuncia \textjapanese{sens\={e}}, non \textjapanese{sensei}. Esistono
        alcuni casi isolati in cui il suono \textjapanese{/e/} si allunga con il kana \textjapanese{え} ed il suono
        \textjapanese{/o/} si allunga con \textjapanese{お}, ma sono molto rari.

        Per il katakana basta aggiungere un trattino dopo il suono vocalico. Ad esempio, \textjapanese{ツアー} si
        pronuncia \textjapanese{tsu\={a}}.

    \section{Suoni aggiuntivi}

        In giapponese non esistono alcuni suoni presenti in altre lingue. Dunque, questi suoni, per quanto possano
        in linea teorica essere scritti anche in hiragana, hanno senso solo in katakana.

        \begin{center}
            \begin{japanese}
                \begin{tabular}{|c|c|c|c|c|c|c|c|}
                    \hline
                    ヴァ va & ワ wa & ファ fa & チャ cha & ダ da & タ ta & ジャ ja & シャ sha\\
                    \hline
                    ヴィ vi & ウィ wi & フィ fi & チ chi & ディ di & ティ ti & ジ ji & シ shi\\
                    \hline
                    ヴ vu & ウ wu & フ fu & チュ chu & ドゥ du & トゥ tu & ジュ ju & シュ shu\\
                    \hline
                    ヴェ ve & ウェ we & フェ fe & チェ che & デ de & テ te & ジェ je & シェ she\\
                    \hline
                    ヴォ vo & ウォ wo & フォ fo & チョ cho & ド do & ト to & ジョ jo & ショ sho\\
                    \hline
                \end{tabular}
            \end{japanese}
        \end{center}

\end{document}