\documentclass{article}

\usepackage{polyglossia}
\setmainlanguage{italian}
\setotherlanguage{japanese}
\newfontfamily\japanesefont{BIZ UDGothic}

\usepackage[landscape, margin=1in]{geometry}

\usepackage{multirow}
\renewcommand{\arraystretch}{1.5}

\title{Hiragana e katakana}
\author{Mattia Martelli}

\begin{document}

    \maketitle

    \tableofcontents

    \section{Hiragana e katakana}

        \begin{center}
            \begin{japanese}
                \begin{tabular}{|c|c|c|c|c|c|c|c|c|c|c|}
                    \hline
                    \multirow{5}{*}{ん ン n}
                        & わ ワ wa & ら ラ ra & や ヤ ya & ま マ ma & は ハ ha & な ナ na & た タ ta & さ サ sa & か カ ka & あ ア a\\
                        \cline{2-11}
                        & ゐ ヰ wi & り リ ri & & み 三 mi & ひ ヒ hi & に 二 ni & ち チ chi & し シ shi & き キ ki & い イ i\\
                        \cline{2-11}
                        & & る ル ru & ゆ ユ yu & む ム mu & ふ フ fu & ぬ ヌ nu & つ ツ tsu & す ス su & く ク ku & う ウ u\\
                        \cline{2-11}
                        & ゑ ヱ we & れ レ re & & め メ me & へ ヘ he & ね ネ ne & て テ te & せ セ se & け ケ ke & え エ e\\
                        \cline{2-11}
                        & を ヲ wo & ろ ロ ro & よ ヨ yo & も モ mo & ほ ホ ho & の ノ no & と ト to & そ ソ so & こ コ ko & お オ o\\
                    \hline
                \end{tabular}
            \end{japanese}
        \end{center}

        \textbf{Nota}: i kana relativi ai suoni \textjapanese{wi} e \textjapanese{we} sono oramai caduti in disuso nell'uso comune.
    
    \section{Dakuten e handakuten}

        \begin{center}
            \begin{japanese}
                \begin{tabular}{|c|c|c|c|c|}
                    \hline
                    ぱ パ pa & ば バ ba & だ ダ da & ざ ザ za & が ガ ga\\
                    \hline
                    ぴ ピ pi & ば バ bi & ぢ ヂ ji & じ ジ ji & ぎ ギ gi\\
                    \hline
                    ぷ プ pu & ぶ ブ bu & づ ヅ zu & ず ズ zu & ぐ グ gu\\
                    \hline
                    ぺ ペ pe & べ べ be & で デ de & ぜ ゼ ze & げ ゲ ge\\
                    \hline
                    ぽ ポ po & ぼ ボ bo & ど ド do & ぞ ゾ zo & ご ゴ go\\
                    \hline
                \end{tabular}
            \end{japanese}
        \end{center}

        \textbf{Nota}: \textjapanese{ぢ} e \textjapanese{づ} possono essere scritti in rōmaji anche come \textjapanese{dji} e \textjapanese{dzi}, rispettivamente.

    \section{Yōon}

        

\end{document}