\documentclass{article}

\usepackage{polyglossia}
\setmainlanguage{italian}
\setotherlanguage{japanese}
\newfontfamily\japanesefont{BIZ UDGothic}

\usepackage[landscape, margin=1in]{geometry}

\usepackage{multirow}
\renewcommand{\arraystretch}{1.5}

\title{Hiragana e katakana}
\author{Mattia Martelli}

\begin{document}

    \maketitle

    \tableofcontents

    \section{Hiragana e katakana}

        \centering
        \begin{japanese}
            \begin{tabular}{ | c | c | c | c | c | c | c | c | c | c | c | }
                \hline
                \multirow{5}{*}{ん ン n}
                    & わ ワ wa & ら ラ ra & や ヤ ya & ま マ ma & は ハ ha & な ナ na & た タ ta & さ サ sa & か カ ka & あ ア a\\
                    \cline{2-11}
                    & ゐ ヰ wi & り リ ri & & み 三 mi & ひ ヒ hi & に 二 ni & ち チ chi & し シ shi & き キ ki & い イ i\\
                    \cline{2-11}
                    & & る ル ru & ゆ ユ yu & む ム mu & ふ フ fu & ぬ ヌ nu & つ ツ tsu & す ス su & く ク ku & う ウ u\\
                    \cline{2-11}
                    & ゑ ヱ we & れ レ re & & め メ me & へ ヘ he & ね ネ ne & て テ te & せ セ se & け ケ ke & え エ e\\
                    \cline{2-11}
                    & を ヲ wo & ろ ロ ro & よ ヨ yo & も モ mo & ほ ホ ho & の ノ no & と ト to & そ ソ so & こ コ ko & お オ o\\
                \hline
            \end{tabular}
        \end{japanese}

        \bigskip

        \textbf{Nota}: i kana relativi ai suoni \textjapanese{wi} e \textjapanese{we} sono oramai caduti in disuso nell'uso comune.

\end{document}