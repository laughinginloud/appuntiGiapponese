\documentclass{article}

\usepackage{polyglossia}
\setmainlanguage{italian}
\setotherlanguage{japanese}
\newfontfamily\japanesefont{BIZ UDGothic}

\usepackage[margin=1in]{geometry}

\usepackage{multirow, longtable}
\renewcommand{\arraystretch}{1.5}

\title{Kanji}
\author{Mattia Martelli}
\date{}

\usepackage{hyperref}
\hypersetup{
    unicode=true,
    bookmarksnumbered=true,
    bookmarksopen=false,
    hidelinks,
    pdfauthor={Mattia Martelli},
    pdftitle={Kanji}
}

\let\ita\textitalian
\newcommand{\itabf}[1]{\ita{\textbf{#1}}}

\begin{document}
    
    \maketitle

    In alcuni casi sono inclusi nelle pronuncie anche i kana aggiuntivi utili a dare il significato specificato.
    Questo è detto okurigana.

    \begin{center}
        \begin{japanese}
            \begin{longtable}{|c|c|c|c|c|}
                \hline
                \itabf{Kanji} & \itabf{Kun-yomi} & \itabf{On-yomi} & \itabf{Significato}\\
                \hline\hline
                人 & ひと & ジン & \ita{Persona}\\
                \hline
                日 & / & ニ & \ita{Sole, giorno}\\
                \hline
                本 & / & ホン & \ita{Origine, libro}\\
                \hline
                学 & / & ガク & \ita{Accademico}\\
                \hline
                先 & / & セン & \ita{Avanti, precedenza}\\
                \hline
                生 & / & セイ & \ita{Vita}\\
                \hline
                高 & たか・い & コウ & \ita{Alto, costoso}\\
                \hline
                校 & / & コウ & \ita{Scuola}\\
                \hline
                小 & ちい・さい & ショウ & \ita{Piccolo}\\
                \hline
                中 & / & チュウ & \ita{Medio, dentro}\\
                \hline
                大 & おお・きい & ダイ & \ita{Grande}\\
                \hline
                国 & くに & コク & \ita{Paese}\\
                \hline
                英 & / & エイ & \ita{Inghilterra}\\
                \hline
                語 & / & ゴ & \ita{Lingua}\\
                \hline
                暑 & あつ・い & / & \ita{Caldo (solo clima)}\\
                \hline
                熱 & あつ・い、ねつ & / & \ita{Calore, febbre}\\
                \hline
            \end{longtable}
        \end{japanese}
    \end{center}

\end{document}