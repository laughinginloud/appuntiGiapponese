\documentclass{article}

\usepackage{polyglossia}
\setmainlanguage{italian}
\setotherlanguage{japanese}
\newfontfamily\japanesefont{BIZ UDGothic}

\usepackage[margin=1in]{geometry}

\usepackage{multirow,longtable}
\renewcommand{\arraystretch}{1.5}

\title{Numeri}
\author{Mattia Martelli}
\date{}

\usepackage{hyperref}
\hypersetup{
    unicode=true,
    bookmarksnumbered=true,
    bookmarksopen=false,
    hidelinks,
    pdfauthor={Mattia Martelli},
    pdftitle={Numeri}
}

\let\ita\textitalian
\newcommand{\itabf}[1]{\ita{\textbf{#1}}}
\newenvironment{tabJap}{\begin{center}\begin{japanese}\begin{longtable}{|c|c|c|}}{\end{longtable}\end{japanese}\end{center}}

\begin{document}

    \maketitle

    \tableofcontents

    \section{Da 1 a 10}

        Per alcuni numeri esistono varie pronuncie. Quelle in grassetto sono quelle più utilizzate.

        \begin{tabJap}
            \hline
            \itabf{Numero} & \itabf{Kanji} & \itabf{Pronuncia}\\
            \hline\hline
            1 & 一 & いち\\
            \hline
            2 & ニ & に\\
            \hline
            3 & 三 & さん\\
            \hline
            4 & 四 & し、\textbf{よん}\\
            \hline
            5 & 五 & ご\\
            \hline
            6 & 六 & ろく\\
            \hline
            7 & 七 & しち、\textbf{なな}\\
            \hline
            8 & 八 & はち\\
            \hline
            9 & 九 & きゅう\\
            \hline
            10 & 十 & じゅう\\
            \hline
        \end{tabJap}

    \newpage

    \section{Da 11 a 99}

        Per contare da 11 fino a 99 basta usare il kanji \textjapanese{十} come contatore.
        Esempi:

        \begin{tabJap}
            \hline
            \itabf{Numero} & \itabf{Scrittura} & \itabf{Pronuncia}\\
            \hline\hline
            11 & 十一 & じゅう・いち\\
            \hline
            20 & 二十 & に・じゅう\\
            \hline
            21 & 二十一 & に・じゅう・いち\\
            \hline
            39 & 三十九 & さん・じゅう・きゅう\\
            \hline
            40 & 四十 & よん・じゅう\\
            \hline
            74 & 七十四 & なな・じゅう・よん\\
            \hline
            99 & 九十九 & きゅう・じゅう・きゅう\\
            \hline
        \end{tabJap}

    \section{Età}

        Per indicare l'età si aggiunge alla fine il kanji \textjapanese{歳}, pronunciato \textjapanese{さい}, che fa
        da contatore. Alle volte si scrive il kanji \textjapanese{才}, in quanto più semplice, anche se è un carattere
        completamente diverso. Esempi:

        \begin{tabJap}
            \hline
            \itabf{Età} & \itabf{Scrittura} & \itabf{Pronuncia}\\
            \hline\hline
            21 & 二十一歳 & に・じゅう・いっ・さい\\
            \hline
            48 & 四十八歳 & よん・じゅう・はっ・さい\\
            \hline
            70 & 七十歳 & なな・じゅっ・さい\\
            \hline\hline
            \multicolumn{3}{|c|}{\itabf{Forme irregolari}}\\
            \hline\hline
            1 & 一歳 & いっ・さい\\
            \hline
            8 & 八歳 & はっ・さい\\
            \hline
            10 & 十歳 & じゅっ・さい\\
            \hline
            20 & 二十歳 & はたち\\
            \hline
        \end{tabJap}
        
\end{document}